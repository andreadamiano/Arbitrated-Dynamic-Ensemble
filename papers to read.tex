STACKING
When it comes to selecting a meta-model for ensemble learning, several research papers and resources can provide valuable insights and methodologies. Here are some notable papers that discuss various strategies for meta-model selection, stacking, and ensemble learning in general:

1. Stacked Generalization
Title: "Stacked Generalization"
Authors: David H. Wolpert
Link: Stacked Generalization (Wolpert, 1992)
Summary: This foundational paper introduces the concept of stacked generalization (stacking) and discusses how to combine multiple models to improve predictive performance. It provides theoretical insights that can guide the selection of meta-models.


2. Ensemble Methods in Machine Learning
Title: "Ensemble Methods in Machine Learning"
Authors: Zhi-Hua Zhou
Link: Ensemble Methods in Machine Learning (Zhou, 2012)
Summary: This paper reviews various ensemble methods, including bagging, boosting, and stacking. It discusses the advantages and challenges of different meta-models within these contexts.


3. Meta-Learning for Ensemble Learning
Title: "Meta-Learning in a Nutshell"
Authors: S. S. M. K. D. Alzubaidi et al.
Link: Meta-Learning in a Nutshell
Summary: This paper explores meta-learning techniques, which can be applied to select the best meta-model based on prior learning experiences. It provides insights into how previous models can inform the choice of the meta-model in new contexts.


4. A Survey on Ensemble Learning
Title: "A Survey on Ensemble Learning"
Authors: Ganaie, M. A., et al.
Link: A Survey on Ensemble Learning
Summary: This survey covers various ensemble learning techniques, including model selection and evaluation strategies, which are crucial for determining an effective meta-model.


5. Stacking and Meta-Modeling
Title: "Learning to Combine: A New Approach to Stacking"
Authors: H. J. van der Laan et al.
Link: Learning to Combine (van der Laan et al., 2007)
Summary: This paper discusses a new approach to stacking that focuses on learning how to optimally combine predictions from different models, emphasizing the importance of choosing the right meta-model.


6. On Stacking and Meta-Learning
Title: "A Comparison of Stacking Methods in the Context of Classification and Regression"
Authors: R. Polikar
Link: A Comparison of Stacking Methods
Summary: This paper compares various stacking methods and discusses the implications of different meta-model choices for classification and regression tasks.


7. Theoretical Perspectives on Stacking
Title: "Theoretical Analysis of Stacking"
Authors: J. H. Friedman
Link: Theoretical Analysis of Stacking
Summary: This paper provides a theoretical framework for understanding stacking and the conditions under which different meta-models can be effective.


Additional Resources
Books on Ensemble Learning: Consider checking out books like "Ensemble Methods: A Unified Approach to Data Mining" by Zhi-Hua Zhou, which provides a comprehensive overview of ensemble techniques, including meta-model selection strategies.
Online Courses and Tutorials: Platforms like Coursera and edX often offer courses on machine learning and ensemble methods that can provide practical guidance on model selection.
Conclusion
These resources will help you gain a deeper understanding of how to choose and evaluate meta-models in stacking and ensemble learning. Exploring these papers will provide theoretical insights, practical examples, and comparative analyses to inform your approach to selecting the best meta-model for your specific tasks.