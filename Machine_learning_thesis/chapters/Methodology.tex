\chapter{Methodology}

\section{Data Preprocessing and Exploratory Data Analysis} % Section 4.1
\subsection{Handling Non-Stationarity with Advanced Transformations} % Subsection 4.1.1
% Content for handling non-stationarity

\subsection{Feature Engineering and Temporal Feature Extraction} % Subsection 4.1.2
% Content for feature engineering

\subsection{Change Point Detection and Anomaly Handling} % Subsection 4.1.3
% Content for change point detection

\section{Model Selection} % Section 4.2
\subsection{Classical Models: Expanding ARIMA/SARIMA with Volatility Modeling (GARCH)} % Subsection 4.2.1
% Content for classical models

\subsection{Machine Learning Models: Leveraging Random Forest, XGBoost, LightGBM for Feature Importance} % Subsection 4.2.2
% Content for machine learning models

\subsection{Deep Learning Models: Hybrid Networks with CNN, LSTM, GRU, and Attention Mechanisms} % Subsection 4.2.3
% Content for deep learning models

\subsection{Incorporating Transformer Models for Long-term Dependencies} % Subsection 4.2.4
% Content for transformer models

\section{Model Architecture Design} % Section 4.3
\subsection{Design of a Multi-branch Hybrid Ensemble (CNN-LSTM-GRU with ARIMA/SARIMA Integration)} % Subsection 4.3.1
% Content for multi-branch hybrid ensemble design

\subsection{Using Attention Mechanisms and Transformers in Time Series} % Subsection 4.3.2
% Content for attention mechanisms and transformers

\subsection{Adaptive Learning in Hybrid Ensembles} % Subsection 4.3.3
% Content for adaptive learning

\section{Model Training and Validation} % Section 4.4
\subsection{Performance Metrics (RMSE, MSE, MAE, MAPE)} % Subsection 4.4.1
% Content for performance metrics

\subsection{Cross-validation and Walk-forward Validation for Time Series} % Subsection 4.4.2
% Content for cross-validation

\subsection{Ensemble Residual Analysis and Diagnostic Testing} % Subsection 4.4.3
% Content for residual analysis and testing