\documentclass{Configuration_gigi/PoliMi3i_thesis}
% CONFIGURATIONS
\usepackage{parskip} % For paragraph layout
\usepackage{setspace} % For using single or double spacing
\usepackage{emptypage} % To insert empty pages
\usepackage{multicol} % To write in multiple columns (executive summary)
\setlength\columnsep{15pt} % Column separation in executive summary
\setlength\parindent{0pt} % Indentation
\raggedbottom  

% PACKAGES FOR TITLES
\usepackage{titlesec}
% \titlespacing{\section}{left spacing}{before spacing}{after spacing}
\titlespacing{\section}{0pt}{3.3ex}{2ex}
\titlespacing{\subsection}{0pt}{3.3ex}{1.65ex}
\titlespacing{\subsubsection}{0pt}{3.3ex}{1ex}
\usepackage{color}

% PACKAGES FOR LANGUAGE AND FONT
\usepackage[english]{babel} % The document is in English  
\usepackage[utf8]{inputenc} % UTF8 encoding
\usepackage[T1]{fontenc} % Font encoding
\usepackage[11pt]{moresize} % Big fonts

% PACKAGES FOR IMAGES
\usepackage{graphicx}
\usepackage{transparent} % Enables transparent images
\usepackage{eso-pic} % For the background picture on the title page
\usepackage{subfig} % Numbered and caption subfigures using \subfloat.
\usepackage{tikz} % A package for high-quality hand-made figures.
\usetikzlibrary{}
\graphicspath{{./Images/}} % Directory of the images
\usepackage{caption} % Coloured captions
\usepackage{xcolor} % Coloured captions
\usepackage{amsthm,thmtools,xcolor} % Coloured "Theorem"
\usepackage{float}

% STANDARD MATH PACKAGES
\usepackage{amsmath}
\usepackage{amsthm}
\usepackage{amssymb}
\usepackage{amsfonts}
\usepackage{bm}
\usepackage[overload]{empheq} % For braced-style systems of equations.
\usepackage{fix-cm} % To override original LaTeX restrictions on sizes

% PACKAGES FOR TABLES
\usepackage{tabularx}
\usepackage{longtable} % Tables that can span several pages
\usepackage{colortbl}

% PACKAGES FOR ALGORITHMS (PSEUDO-CODE)
\usepackage{algorithm}
\usepackage{algorithmic}

% PACKAGES FOR REFERENCES & BIBLIOGRAPHY
\usepackage[colorlinks=true,linkcolor=black,anchorcolor=black,citecolor=black,filecolor=black,menucolor=black,runcolor=black,urlcolor=black]{hyperref} % Adds clickable links at references
\usepackage{cleveref}
\usepackage[square, numbers, sort&compress]{natbib} % Square brackets, citing references with numbers, citations sorted by appearance in the text and compressed
\bibliographystyle{abbrvnat} % You may use a different style adapted to your field

% OTHER PACKAGES
\usepackage{pdfpages} % To include a pdf file
\usepackage{afterpage}
\usepackage{lipsum} % DUMMY PACKAGE
\usepackage{fancyhdr} % For the headers
\fancyhf{}

% Input of configuration file. Do not change config.tex file unless you really know what you are doing. 
% Define blue color typical of polimi
\definecolor{bluepoli}{cmyk}{0.4,0.1,0,0.4}

% Custom theorem environments
\declaretheoremstyle[
  headfont=\color{bluepoli}\normalfont\bfseries,
  bodyfont=\color{black}\normalfont\itshape,
]{colored}

% Set-up caption colors
\captionsetup[figure]{labelfont={color=bluepoli}} % Set colour of the captions
\captionsetup[table]{labelfont={color=bluepoli}} % Set colour of the captions
\captionsetup[algorithm]{labelfont={color=bluepoli}} % Set colour of the captions

\theoremstyle{colored}
\newtheorem{theorem}{Theorem}[chapter]
\newtheorem{proposition}{Proposition}[chapter]

% Enhances the features of the standard "table" and "tabular" environments.
\newcommand\T{\rule{0pt}{2.6ex}}
\newcommand\B{\rule[-1.2ex]{0pt}{0pt}}

% Pseudo-code algorithm descriptions.
\newcounter{algsubstate}
\renewcommand{\thealgsubstate}{\alph{algsubstate}}
\newenvironment{algsubstates}
  {\setcounter{algsubstate}{0}%
   \renewcommand{\STATE}{%
     \stepcounter{algsubstate}%
     \Statex {\small\thealgsubstate:}\space}}
  {}

% New font size
\newcommand\numfontsize{\@setfontsize\Huge{200}{60}}

% Title format: chapter
\titleformat{\chapter}[hang]{
\fontsize{50}{20}\selectfont\bfseries\filright}{\textcolor{bluepoli} \thechapter\hsp\hspace{2mm}\textcolor{bluepoli}{|   }\hsp}{0pt}{\huge\bfseries \textcolor{bluepoli}
}

% Title format: section
\titleformat{\section}
{\color{bluepoli}\normalfont\Large\bfseries}
{\color{bluepoli}\thesection.}{1em}{}

% Title format: subsection
\titleformat{\subsection}
{\color{bluepoli}\normalfont\large\bfseries}
{\color{bluepoli}\thesubsection.}{1em}{}

% Title format: subsubsection
\titleformat{\subsubsection}
{\color{bluepoli}\normalfont\large\bfseries}
{\color{bluepoli}\thesubsubsection.}{1em}{}

% Shortening for setting no horizontal-spacing
\newcommand{\hsp}{\hspace{0pt}}

\makeatletter
% Renewcommand: cleardoublepage including the background pic
\renewcommand*\cleardoublepage{%
  \clearpage\if@twoside\ifodd\c@page\else
  \null
  \AddToShipoutPicture*{\BackgroundPic}
  \thispagestyle{empty}%
  \newpage
  \if@twocolumn\hbox{}\newpage\fi\fi\fi}
\makeatother

%For correctly numbering algorithms
\numberwithin{algorithm}{chapter}

\begin{document}

\chapter{Missions to Small Bodies}\label{Ch:Missions to Small Bodies}

Since the discovery of Ceres in 1801, the primary method for studying small bodies of the Solar System has been through \textbf{telescopic characterization}, using ground-based optical observations. This technique provides critical information about the target's orbit and some physical properties. Moreover, continuous observation over time allows scientists to construct the body's \textbf{lightcurve} -- the variation in the intensity of light reflected by the target as function of time. The lightcurve is valuable for drawing conclusions, more or less accurately, about the target's physical properties, such as its shape and spin state. Additionally, by studying the \textbf{electromagnetic spectrum} of the radiation emitted or reflected by the body, researchers can infer its possible composition. 

However, the results from telescopic characterization are often subject to uncertainties, which increase with the target's distance from Earth and decrease with its size. As a result, characterizing distant or smaller bodies accurately becomes challenging. A potential solution to this limitation is the use of \textbf{space telescopes}, like \textbf{NEOWISE} \cite{neowise} and the upcoming \textbf{NEO Surveyor} \cite{neosurveyor}. While these tools can improve our knowledge of NEAs, especially in detecting and observing small ones, they still fall short in providing a complete and precise characterization of some targets, particularly the smaller NEAs.

At this point, \textbf{robotic space exploration missions} play a critical role. These missions allow spacecraft to visit multiple small bodies in succession or to rendezvous with a specific target. Through in situ physical characterization and, when possible, the return of surface regolith samples, these missions offer a deeper and more accurate understanding of the target than can be achieved through telescopic observation alone.


\section{First Encounters with Small Bodies}\label{Sec:First Encounters with Small Bodies}

The first encounter of a man-made object with a small body occurred in 1985, when NASA's \textbf{International Cometary Explorer} (\textbf{ICE}, \cite{ice}) spacecraft became the first to fly past a comet, \textbf{21P/Giacobini-Zinner}. This milestone was followed by a series of missions in 1986, where spacecraft from different space agencies performed fast fly-bys of comet \textbf{1P/Halley}. These missions included \textbf{VEGA-1} and \textbf{VEGA-2} \cite{vega}, \textbf{Suisei} and \textbf{Sakigate} \cite{SuiseiSakigate}, and \textbf{Giotto} \cite{giotto}. Among them, Giotto returned exceptionally valuable scientific results, particularly concerning the structure of the comet’s nucleus. Additionally, Giotto's extended mission provided the opportunity to encounter another small body, the comet \textbf{26P/Grigg-Skjellerup}, in 1992. 

In 1991, the first asteroid encounter occurred when NASA's \textbf{Galileo} spacecraft \cite{galileo}, on its way to Jupiter, flew past \textbf{951 Gaspra} in the Main Asteroid Belt. Two years later, Galileo made another historic encounter always in the Main Belt, this time with asteroid \textbf{243 Ida} and its moon Dactyl, marking the first discovery of a moon orbiting an asteroid.


\section{NEAR Mission}\label{Sec:NEAR Mission}

In 1995, NASA launched the \textbf{Near Earth Asteroid Rendezvous} (\textbf{NEAR}, \cite{near}) mission, the first spacecraft specifically designed for the exploration of a NEA. NEAR made history in 2000 by becoming the first spacecraft to rendezvous with a small Solar System body, asteroid \textbf{433 Eros}, orbiting it for about a year before achieving a soft landing on its surface. During its deep-space cruise, NEAR also flew past asteroid \textbf{253 Mathilde} in 1997, gathering additional scientific data.

The flight dynamics of NEAR's asteroid approach and proximity phases are detailed in \cite{near}, while its mission analysis is discussed in \cite{nearMA}. The asteroid \textbf{Approach Phase} was achieved using a DSM and several TCMs to gradually reduce the spacecraft’s relative position and velocity with respect to Eros. This culminated in insertion into an elliptical orbit around the asteroid through via a hyperbolic trajectory. 

The subsequent \textbf{Close Proximity Phase} focused on characterizing Eros' Northern Hemisphere, which was sunlit at the arrival epoch. The strategy involved progressively decreasing retrograde circular orbits, from 24 down to 6 asteroid's radii, using a descent-circularize method (Figure~\ref{fig:near1}). When favourable illumination conditions were achieved, the Equatorial and Southern Hemispheres were characterized using even lower-altitude orbits, descending from 6 to 2.4 asteroid radii (Figure~\ref{fig:near2} shows one of these orbits). The mission concluded with an attempt of controlled descent and soft landing on Eros' surface, accomplished through four final maneuvers.

\begin{figure}[H]
    \centering
    \subfloat[Northern hemisphere characterization \cite{near}.\label{fig:near1}]{
        \includegraphics[scale=0.63]{Images/near1.jpg}
    }
    \quad
    \subfloat[Low-altitude southern hemisphere fly-over \cite{near}.\label{fig:near2}]{
        \includegraphics[scale=0.67]{Images/near2.jpg}
    }
    \caption[NEAR proximity operations.]{NEAR close proximity strategy.}
    \label{fig:near}
\end{figure}

In the meantime, NASA launched other minor missions to perform fly-bys with multiple targets or to carry out impact experiments. The \textbf{Deep Space 1} mission, a low-thrust technology validation mission, passed by asteroid \textbf{9969 Braille} in 1999 and comet \textbf{19P/Borrelly} in 2001 \cite{deepspace1}. The \textbf{Stardust} mission flew past asteroid \textbf{5535 Annefrank} in 2002, followed by a close encounter with comet \textbf{81P/Wild 2} in 2004, where it collected dust and volatile samples from the comet's coma \cite{stardust}. The \textbf{Deep Impact} mission released an impactor that collided with comet \textbf{9P/Tempel 1}, capturing images before and after the impact in 2005 \cite{deepimpact}; its extended mission, \textbf{EPOXI}, later flew past comet \textbf{103P/Hartley 2} \cite{epoxi}.


\section{Rosetta Mission}\label{Sec:Rosetta Mission}

In 2004, the \textbf{Rosetta} mission \cite{rosetta}, a collaborative effort involving ESA, NASA and several European national space agencies, was launched with the ambitious goal of rendezvousing with comet \textbf{P67/Churyumov-Gerasimenko} in 2014. Rosetta became the first spacecraft to rendezvous with a comet, orbit it, and release a lander, named \textbf{Philae}, onto its surface. This mission was incredibly successful, captivating global media and public attention, providing a significant boost to the exploration of small bodies in the Solar System. Despite the complex and hazardous environment near the comet, characterized by ejecta from the nucleus and the presence of the cometary tail, Rosetta successfully approached the comet, orbited it, mapped its surface and collected significant scientific data from both orbit and the comet's surface with the help of lander Philae. 

Moreover, during its 10-years interplanetary cruise, Rosetta flew past two Main Belt's asteroids: \textbf{2867 Steins} in 2008 \cite{rosetta_steins} and \textbf{21 Lutetia} in 2010 \cite{rosetta_lutetia}, providing valuable data for the physical and photometric characterization of both asteroids.

Rosetta's trajectory in the vicinity of the comet is described in detail in \cite{rosettaFD}. The \textbf{Approach} to the comet followed a waypoint strategy, gradually reducing the spacecraft’s distance and velocity relative to the comet over three main phases:

\begin{itemize}
    \item \textbf{Near Comet Drift} (NCD, Figure~\ref{fig:rosetta1}): Involves four manoeuvres targeting the Comet Acquisition Point (CAP), where the first images of the comet were captured by the spacecraft's instruments.
    \item \textbf{Far Approach Trajectory} (FAT, Figure~\ref{fig:rosetta2}): Consisted of four manoeuvres to reach the Approach Transition Point (ATP), located 1000 comet radii from the nucleus.
    \item \textbf{Close Approach Trajectory} (CAT, Figure~\ref{fig:rosetta3}): Involves three manoeuvres to reach the Orbit Insertion Point (OIP), 40 comet radii from the nucleus, where the spacecraft executed an orbit insertion manoeuvre via a hyperbolic trajectory.
\end{itemize}

\begin{figure}[H]
    \centering
    \subfloat[NCD strategy \cite{rosettaFD}.\label{fig:rosetta1}]{
        \includegraphics[scale=0.5]{Images/rosetta1.jpg}
    }
    %\quad
    \subfloat[FAT strategy \cite{rosettaFD}.\label{fig:rosetta2}]{
        \includegraphics[scale=0.4]{Images/rosetta2.jpg}
    }
    %\quad
    \subfloat[CAT strategy \cite{rosettaFD}.\label{fig:rosetta3}]{
        \includegraphics[scale=0.6]{Images/rosetta3.jpg}
    }
    \caption[Rosetta approach phase.]{Rosetta mission approach strategy.}
    \label{fig:rosettaAPP}
\end{figure}

The \textbf{Close Proximity Phase} began with the \textbf{Transition to Global Mapping} (TGM, Figure~\ref{fig:rosetta4}). This phase involved a series of manoeuvres, starting with the circularization of the spacecraft's trajectory at the pericentre of the incoming hyperbola. The spacecraft then transitioned into a Homann transfer orbit, followed by a final circularization manoeuvre that placed it into its mapping orbit. Following the TGM phase, the mission entered the scientific observation phase, which consisted of two parts: the \textbf{Global Mapping Phase} (GMP, Figure~\ref{fig:rosetta5}) and the \textbf{Close Observation Phase} (COP, Figure~\ref{fig:rosetta6}). During GMP the spacecraft aimed to map at least 80\% of the comet surface from a polar retrograde orbit at a distance in the range from 10 to 25 comet radii. Once the mapping requirements were met, the COP began, consisting of a series of fly-overs at less than 1 comet radius, over five different regions of the comet. Precise phasing maneuvers ensured the spacecraft achieved the correct illumination and distance for each flyover.

\begin{figure}[H]
    \centering
    \subfloat[TMG strategy \cite{rosettaFD}.\label{fig:rosetta4}]{
        \includegraphics[scale=0.7]{Images/rosetta4bis.jpg}
    }
    %\quad
    \subfloat[GMP strategy \cite{rosettaFD}.\label{fig:rosetta5}]{
        \includegraphics[scale=0.75]{Images/rosetta5bis.jpg}
    }
    %\quad
    \subfloat[COP strategy \cite{rosettaFD}.\label{fig:rosetta6}]{
        \includegraphics[scale=0.63]{Images/rosetta6bis.jpg}
    }
    \caption[Rosetta proximity operations.]{Rosetta mission strategy for proximity operations.}
    \label{fig:rosettaCPO}
\end{figure}


\section{Hayabusa and Hayabusa2 Missions}\label{Sec:Hayabusa and Hayabusa2 Missions}

The Japan Aerospace Exploration Agency (JAXA) made significant contributions to the exploration of NEAs by pioneering sample return missions. JAXA launched two highly successful missions: \textbf{Hayabusa} (previously named MUSES-C) \cite{hayabusa} in 2003 and \textbf{Hayabusa2} \cite{hayabusa2} in 2014. These missions returned surface samples from asteroids \textbf{25143 Itokawa} and \textbf{162173 Ryugu}, respectively. Recently, Hayabusa2 mission was extended to encounter asteroid \textbf{(98943) 2001 CC21} in 2026, and to rendezvous with \textbf{1998 KY26} in 2031, performing a 10-year deep-space trajectory \cite{hayabusa2ext}. 

Hayabusa and Hayabusa2 employed a different strategy compared to previous missions, as discussed in \cite{hayabusaGNC} and \cite{hayabusa2MA}. After rendezvousing with the target asteroid, the spacecraft mapped its surface using a \textbf{hovering strategy}, which took advantage of the natural spin of the target to observe its surface from a fixed position (see Figure~\ref{fig:hayabusa}). This strategy required minimal station-keeping manoeuvres. While this method was less efficient in terms of surface coverage, it allowed for significant propellant savings, crucial for performing delicate, autonomous Touch-And-Go (TAG) sample collection maneuvers. 

Moreover, Hayabusa was a technological demonstrator for key space exploration technologies, including autonomous navigation and optical guidance systems \cite{hayabusaGNC}, which were vital for the success of surface regolith sample collection. The success of this mission paved the way for subsequent asteroid sample return missions, demanstrating the feasibility of sample return through autonomous touch-and-go maneuvers, the same solution adopted by successively by Hayabusa2 and  OSIRIS-REx missions.

\begin{figure}[H]
    \centering
    \includegraphics[width=0.65\textwidth]{Images/hayabusa.jpg}
    \caption[Hayabusa approach and landing.]{Hayabusa approaching and landing scenario \cite{hayabusaGNC}.}
    \label{fig:hayabusa}
\end{figure}


\section{OSIRIS-REx Mission}\label{Sec:OSIRIS-REx Mission}

In 2016, two years after the launch of Hayabusa2, NASA launched its most ambitious mission to a NEA: the \textbf{Origins Spectral Interpretation Resource Identification Security Regolith Explorer} (\textbf{OSIRIS-REx}, \cite{osirisrex}). The mission aimed to rendezvous with asteroid \textbf{101955 Bennu} in 2018 and return at least 60 grams of surface regolith samples to Earth in 2023. This mission represents the state-of-the-art in asteroid sample return, as it successfully mapped 100\% of Bennu's surface, conducted thorough scientific operations \cite{osirisrexDTM}, and autonomously performed a TAG manoeuvre to collect surface samples \cite{osirisrexTAG}, returning them safely to Earth. Following the return in Earth's orbit, it was revealed that the spacecraft had enough remaining propellant to extend its mission. The extended mission, named \textbf{OSIRIS-APEx}, will rendezvous with asteroid \textbf{99942 Apophis} in 2029, during its closest approach to Earth \cite{osirisapex}.

The trajectory followed by OSIRIS-REx during approach and proximity operations is extensively detailed in \cite{osirisrex} and \cite{osirisrexFD}. The mission's \textbf{Approach Phase} was relatively straightforward, involving three deterministic approach manoeuvres to position the spacecraft in the starting point for the proximity operations, allowing sufficient time to optically acquire the target. On the other hand, \textbf{Proximity Operations} were far more complex, comprising five distinct phases that combined closed orbits and hyperbolic trajectories. 

The first phase, known as the \textbf{Preliminary Survey} (Figure~\ref{fig:osirisrex1}), involved three hyperbolic trajectories passing over Bennu's North Pole, Equator, and South Pole in succession. A total of five manoeuvres was performed during this phase. Preliminary Survey was followed by the \textbf{Orbit A} (Figure~\ref{fig:osirisrex2}), a Sun-Stabilized Terminator Orbit (SSTO) at an altitude of approximately 5 asteroid's radii. This orbit required three manoeuvres for proper insertion. Next, the mission entered the \textbf{Detailed Survey}, which involved a comprehensive scientific observation campaign conducted via two series of hyperbolic passes:

\begin{itemize}
    \item The \textbf{"Baseball Diamond"} (Figure~\ref{fig:osirisrex3}): A series of hyperbolic passes over four sub-spacecraft points near 40° North and South latitude, at 10 a.m. and 2 p.m. local time on the surface, at an altitude of about 13 asteroid's radii.
    \item The \textbf{Equatorial Stations} (Figure~\ref{fig:osirisrex4}): A set of North-South hyperbolic passes over different Equator locations, at an altitude of about 19 asteroid's radii.
\end{itemize}

This intense observation phase is followed by the \textbf{Orbit B} (Figure~\ref{fig:osirisrex5}), another SSTO at an altitude of 3.8 asteroid's radii. This orbit required five manoeuvres and the use of an intermediate orbit for correct insertion. Orbit B serves both as a platform for accurate scientific observations, aiding in the selection of candidate sampling sites, and as the "home base" for subsequent close-range reconnaissance passes. During the final \textbf{Reconnaissance Phase} (Figure~\ref{fig:osirisrex6}), the spacecraft conducted two sets of two hyperbolic fly-bys at altitudes of 0.86 and 2 asteroid's radii to closely examine the primary and secondary candidate sampling site. 

\begin{figure}[H]
    \centering
    \subfloat[Preliminary survey \cite{osirisrex}.\label{fig:osirisrex1}]{
        \includegraphics[scale=0.5]{Images/osirisrex1.jpg}
    }
    %\quad
    \subfloat[Orbit A \cite{osirisrex}.\label{fig:osirisrex2}]{
        \includegraphics[scale=0.5]{Images/osirisrex2.jpg}
    }
    %\quad
    \subfloat[Baseball diamond \cite{osirisrex}.\label{fig:osirisrex3}]{
        \includegraphics[scale=0.3]{Images/osirisrex3.jpg}
    } \newline
    %\quad
    \subfloat[Equatorial passes \cite{osirisrexFD}.\label{fig:osirisrex4}]{
        \includegraphics[scale=0.45]{Images/osirisrex4.jpg}
    }
    %\quad
    \subfloat[Orbit B \cite{osirisrex}.\label{fig:osirisrex5}]{
        \includegraphics[scale=0.35]{Images/osirisrex5.jpg}
    }
    %\quad
    \subfloat[Reconnaissance \cite{osirisrexFD}.\label{fig:osirisrex6}]{
        \includegraphics[scale=0.55]{Images/osirisrex6.jpg}
    }
    \caption[OSIRIS-REX proximity operations.]{OSIRIS-REx strategy for proximity operations.}
    \label{fig:osirisrexCPO}
\end{figure}


\section{Lucy Mission}\label{Sec:Lucy Mission}

Building on the valuable scientific insights gained from asteroid fly-bys, NASA launched the \textbf{Lucy} mission in 2021 with the ambitious goal of exploring five Trojan asteroids and their possible satellites over a 12-years interplanetary mission duration \cite{lucy}. In order of encounter, Lucy will visit the following asteroids:

\begin{itemize}
    \item \textbf{3548 Eurybates} and its small moon Queta (2027)
    \item \textbf{15094 Polymele} and its natural satellite (2027)
    \item \textbf{11351 Leucus} (2028)
    \item \textbf{21900 Orus} (2028)
    \item \textbf{617 Patroclus} and its moon Menoetius (2033)
\end{itemize}

In addition to the Trojan asteroids, Lucy will also fly-by two Main Belt's asteroids, during its interplanetary journey: \textbf{152830 Dinkinesh} (2023) and \textbf{52246 Donaldjohanson} (2025). After completing its nominal mission, Lucy will continue traveling through the Trojan asteroids swarms, with the possibility of encountering additional bodies.


\section{Dawn and Psyche Missions}\label{Sec:Dawn and Psyche Missions}

Solar Electric Propulsion (SEP) is typically not well-suited for asteroid rendezvous missions, given the weak gravitational field that most asteroids exert on the spacecraft. However, in missions targeting asteroids with sufficiently strong gravity field, either due to their large size or high density, SEP becomes a viable option. This is exemplified by NASA's \textbf{Dawn} mission \cite{dawn}, launched in 2007 to reach the heart of the Main Asteroid Belt, where it visited \textbf{4 Vesta} in 2011 and \textbf{1 Ceres} in 2015, the two most massive Main Belt's asteroids. Another mission fully enabled by SEP is NASA’s \textbf{Psyche} spacecraft \cite{psyche}, launched in 2023 to rendezvous with asteroid \textbf{16 Psyche}, the largest known metal asteroid in the Solar System. 

From a mission analysis perspective, the trajectory planning for these SEP missions differs significantly from missions using traditional impulsive propulsion. The electric propulsion system provides continuous, low-thrust acceleration, enabling a different class of orbital trajectories. Unlike impulsive missions, SEP missions are restricted to closed trajectories; hyperbolic loops are excluded due to the continuous and low nature of the thrust. The trajectory design carried out for Psyche mission is exposed in \cite{psycheMA}, while the one done for Dawn mission is described from a flight dynamics perspective in \cite{dawnFD}, concerning the operations at 4 Vesta.

The \textbf{Orbital Operations} for the Psyche spacecraft will begin with a gradual approach to the asteroid, followed by low-thrust transfers into four progressively lower science orbits (Figure~\ref{fig:GG}):
\begin{itemize}
    \item \textbf{Orbit A}: A closed polar orbit at an altitude of 7.2 target's radii, used to determine Psyche's gravity field, which will guide the design of Orbit B and its orbital transfer.
    \item \textbf{Orbit B}: A closed polar orbit at an altitude of 3.6 target's radii, designed to meet imaging requirements and achieve complete topographic mapping of the asteroid's surface.
    \item \textbf{Orbit C}: A closed polar orbit at an altitude of 2.5 target's radii, aimed at completing the gravity estimation, mapping and magnetometry requirements.
    \item \textbf{Orbit D}: A closed retrograde equatorial orbit at an altitude of 1.75 target's radii, dedicated to fulfilling the mission’s core science objectives, particularly determining Psyche's surface composition. 
\end{itemize}

Managing low-thrust orbital transfers is critical due to the longer time required to counter unexpected perturbations, and the potential for orbital instability caused by spin-orbit resonances. Additionally, eclipses represent a mission-critical factor. The spacecraft's power relies on solar energy, and eclipse duration must be limited to prevent battery discharge and loss of thrust. A key requirement for the Psyche mission is to ensure orbital stability while limiting eclipse times to a maximum of 65 minutes. On the other hand, an advantage of low-thrust propulsion is the possibility of recovery at anytime, given the continuous nature of this kind of propulsion.

\begin{figure}[H]
    \centering
    \includegraphics[width=0.725\textwidth]{psyche.jpg}
    \caption[Psyche proximity operations.]{Psyche close proximity trajectory \cite{psycheMA}.}
    \label{fig:GG}
\end{figure}


\section{DART and Hera Missions}\label{Sec:DART and Hera Missions}

A significant aspect of NEAs research is planetary defence. In this context, NASA and ESA are collaborating on the \textbf{AIDA} (Asteroid Impact and Deflection Assessment, \cite{aida}) mission, aimed at understanding how an impactor spacecraft can alter an asteroid's trajectory, in particular its orbital motion. The target for this experiment is Dimorphos, the smaller body of the binary asteroid \textbf{65803 Didymos}.

The AIDA mission involves two spacecrafts, one from NASA and the other from ESA. NASA's spacecraft conducted the \textbf{Double Asteroid Redirection Test} (\textbf{DART}, \cite{dart}), launched in 2021 to impact Dimorphos in 2022 after following a carefully designed interplanetary trajectory \cite{dartMA}. DART's mission was to directly collide with the asteroid to test deflection techniques for planetary defense.

Following this, ESA launched the \textbf{Hera} mission in 2024 to study the impact's effects on Dimorphos \cite{hera}. In 2026, Hera will become the first spacecraft to rendezvous with a binary asteroid system, performing detailed scientific observations of both Didymos and Dimorphos, as well as analyzing the effects of the DART's impact.

Hera's proximity trajectory is described in detail in \cite{hera} and \cite{heraGNC}. The complex dynamics in the vicinity of the binary system, driven by the gravitational interactions between the two bodies, made closed orbits unstable. Instead, the mission planners selected hyperbolic loop trajectories \cite{heraGNC}. The spacecraft's asteroid \textbf{Approach Phase} will involve five manoeuvres, culminating in the orbit insertion, after which the \textbf{Proximity Phase} will begin. This phase will be broken down into four sub-phases. The first will be the \textbf{Early Characterization Phase} (Figure~\ref{fig:hera1}), using hyperbolic arcs to survey the system. This phase will end with the \textbf{Payload Deployment Phase}, where two CubeSats will be released. Following that, the spacecraft will reduce its distance from Didymos, entering the \textbf{Detailed Characterization Phase} (Figure~\ref{fig:hera2}), during which it will perform twelve close hyperbolic fly-bys, gradually refining its understanding of the system and the impact effects. The mission will conclude with the \textbf{Experimental Phase} (Figure~\ref{fig:hera3}), characterized by autonomous hyperbolic fly-bys at extremely low altitudes, designed to test spacecraft capabilities and gather further scientific data.

\begin{figure}[H]
    \centering
    \subfloat[Early characterization \cite{heraGNC}.\label{fig:hera1}]{
        \includegraphics[scale=0.64]{Images/hera1.jpg}
    }
    %\quad
    \subfloat[Detailed characterization \cite{heraGNC}.\label{fig:hera2}]{
        \includegraphics[scale=0.65]{Images/hera2.jpg}
    }
    %\quad
    \subfloat[Experimental phase \cite{heraGNC}.\label{fig:hera3}]{
        \includegraphics[scale=0.53]{Images/hera3.jpg}
    }
    \caption[Hera proximity operations.]{Hera mission strategy for proximity operations.}
    \label{fig:heraCPO}
\end{figure}


\subsection{CubeSat Technology in Asteroid Missions}\label{Ssec:CubeSat Technology in Asteroid Missions}

The DART and Hera missions are also groundbreaking from a technological perspective, as they were the first asteroid missions to incorporate \textbf{CubeSat technology}. The success of CubeSats in various space missions has led to their growing use in supporting future deep-space exploration.

The first asteroid mission aided by CubeSats was DART, which carried the \textbf{Light Italian CubeSat for Imaging of Asteroids} (\textbf{LICIACube}, \cite{liciacube}). LICIACube was released during DART's approach to Didymos system, where it performed a fly-by, capturing images of the impact and providing valuable data to characterize the binary system. 

Similarly, the Hera mission will be assisted by two CubeSats during proximity operations: \textbf{Milani} and \textbf{Juventas} \cite{heracube}. Politecnico di Milano is responsible for Milani's mission analysis and Guidance, Navigation and Control (GNC) subsystem design, with the development described in several publications \cite{milani1}, \cite{milani2}, \cite{milani3}. Due to the complex non-linear dynamics of the binary asteroid system, maintaining stable closed orbit is only feasible for SSTO close to the primary body. Milani will employ hyperbolic loop trajectories (Figure~\ref{fig:milaniCPO}), while Juventas will use SSTO (Figure~\ref{fig:juventasCPO}). A key challenge in the mission involves coordinating the manoeuvres between the Hera spacecraft and the two CubeSats to ensure successful proximity operations. 

Furthermore, is under study at ESA the design of the \textbf{Miniaturised Asteroid Remote Geophysical Observer} (\textbf{M-ARGO}, \href{https://www.esa.int/Enabling_Support/Space_Engineering_Technology/Shaping_the_Future/M-Argo_Journey_of_a_suitcase-sized_asteroid_explorer}{ESA}) mission, which aims to be the first stand-alone interplanetary CubeSat. M-ARGO's goal is to rendezvous with an asteroid with high rotation velocity, demonstrating the capabilities and technologies needed for stand-alone CubeSat in deep-space exploration.

\begin{figure}[H]
    \centering
    \subfloat[Far range phase \cite{milani2}.\label{fig:milani1}]{
        \includegraphics[scale=0.72]{Images/milani2.jpg}
    }
    \quad
    \subfloat[Close range phase \cite{milani2}.\label{fig:milani2}]{
        \includegraphics[scale=0.9]{Images/milani3.jpg}
    }
    \caption[Milani proximity operations.]{Milani close proximity trajectory.}
    \label{fig:milaniCPO}
\end{figure}

\begin{figure}[H]
    \centering
    \subfloat[SSTO around Dydimos \cite{heracube}.\label{fig:juventas1}]{
        \includegraphics[scale=0.85]{Images/juventas3.jpg}
    }
    \quad
    \subfloat[Transfer from SSTO1 to SSTO2 \cite{heracube}.\label{fig:juventas2}]{
        \includegraphics[scale=0.8]{Images/juventas2.jpg}
    }
    \caption[Juventas proximity operations.]{Juventas close proximity trajectory.}
    \label{fig:juventasCPO}
\end{figure}


\section{Missions to Small Bodies: a Summary}\label{Sec:Missions to Small Bodies: a Summary}

As outlined in the previous sections, space missions to small bodies of the Solar System come in various forms. The simplest of these are the \textbf{fly-by} missions, in which a spacecraft makes a close approach to the target, performs scientific observation for a limited time, and then continues its journey. \textbf{Impact} missions follow a similar structure, but involve an impactor -- either a payload or the spacecraft itself -- that must collide with the target. 

The complexity of a mission increases significantly when a \textbf{rendezvous} and the following orbiting phase are incorporated, and even more so when a \textbf{landing} or a \textbf{sample return} is required. It is also important to note that many missions can involve a combination of these elements. For example, a rendezvous mission might include several fly-bys during its interplanetary cruise and could also feature a sample return or landing. Similarly, an impact mission may involve fly-bys of other small bodies during its cruise, as well as a fly-by of the target around the epoch of the impact. 

Another key aspect influencing mission design is the \textbf{propulsion system} used. Chemical and electric propulsion systems impose vastly different constraints on spacecraft design and trajectory planning. Each type presents unique challenges that must be addressed to ensure mission success.

Finally, the rapid advancement and growing success of \textbf{CubeSat technology} are driving its increased use in supporting future missions. While this enhances mission capabilities, it also adds complexity. Coordinating the manoeuvres of the mothercraft and its CubeSats becomes a critical design challenge, requiring careful planning and synchronization.

A summary of the previous missions, highlighting key aspects, is presented in Table~\ref{tab:review}. For each mission, the table includes the mission name, the launch year, the list of visited small bodies, the arrival year for each target, and the type of mission performed during each encounter. This overview emphasizes the significant differences between various missions to small bodies, reflecting the evolution and diversification of exploration strategies since the beginning of Solar System exploration.

\begin{table}[H]
\centering 
    \begin{tabular}{|p{10em} c p{12em} p{4em} p{3em}|}
    \hline 
    \rowcolor{bluepoli!40}
    \textbf{Mission Name} & \textbf{Launch} & \textbf{Target} & \textbf{Arrival} & \textbf{Type} \T\B \\
    \hline \hline
    \textbf{\footnotesize ICE} & \footnotesize 1978 & \footnotesize 21P/Giacobini-Zinner & \footnotesize 1985 & \footnotesize FB \T\B\\
    \hline
    \textbf{\footnotesize VEGA-1, VEGA-2} & \footnotesize 1984 & \footnotesize 1P/Halley & \footnotesize 1986 & \footnotesize FB \T\B\\
    \hline
    \textbf{\footnotesize Sakigake, Suisei} & \footnotesize 1985 & \footnotesize 1P/Halley & \footnotesize 1986 & \footnotesize FB \T\B\\
    \hline
    \textbf{\footnotesize Giotto} & \footnotesize 1985 & \footnotesize 1P/Halley \newline 26P/Grigg-Skjellerup & \footnotesize 1986 \newline 1992 & \footnotesize FB \newline FB \T\B\\
    \hline
    \textbf{\footnotesize Galileo} & \footnotesize 1989 & \footnotesize 951 Gaspra \newline 243 Ida & \footnotesize 1991 \newline 1993 & \footnotesize FB \newline FB \T\B\\
    \hline
    \textbf{\footnotesize NEAR} & \footnotesize 1996 & \footnotesize 253 Mathilde \newline 433 Eros & \footnotesize 1997 \newline 2000 & \footnotesize FB \newline R, L \T\B\\
    \hline
    \textbf{\footnotesize Deep Space 1} & \footnotesize 1998 & \footnotesize 9969 Braille \newline 19P/Borrelly & \footnotesize 1999 \newline 2001 & \footnotesize FB \newline FB \T\B\\
    \hline
    \textbf{\footnotesize Stardust} & \footnotesize 1999 & \footnotesize 5535 Annefrank \newline 81P/Wild 2 \newline 9P/Tempel 1 & \footnotesize 2002 \newline 2004 \newline 2011 & \footnotesize FB \newline SR \newline FB \T\B\\
    \hline
    \textbf{\footnotesize Hayabusa} & \footnotesize 2003 & \footnotesize 25143 Itokawa & \footnotesize 2005 & \footnotesize R, SR \T\B\\
    \hline
    \textbf{\footnotesize Rosetta} & \footnotesize 2004 & \footnotesize 2867 Steins \newline 21 Lutetia \newline 67P/Churyumov-Gerasimenko & \footnotesize 2008 \newline 2010 \newline 2014 & \footnotesize FB \newline FB \newline R, L \T\B\\
    \hline
    \textbf{\footnotesize Deep Impact} & \footnotesize 2005 & \footnotesize 9P/Tempel 1 \newline 103P/Hartley 2 & \footnotesize 2005 \newline 2010 & \footnotesize I, FB \newline FB \T\B\\
    \hline
    \textbf{\footnotesize Dawn} & \footnotesize 2007 & \footnotesize 4 Vesta \newline 1 Ceres & \footnotesize 2011 \newline 2015 & \footnotesize R \newline R \T\B\\
    \hline
    \textbf{\footnotesize Hayabusa2} & \footnotesize 2014 & \footnotesize 162173 Ryugu \newline (98943) 2001 CC21 \newline 1998 KY26 & \footnotesize 2018 \newline 2026 \newline 2031 & \footnotesize R, SR \newline FB \newline FB \T\B\\
    \hline
    \textbf{\footnotesize OSIRIS-REx} & \footnotesize 2016 & \footnotesize 101955 Bennu \newline 99942 Apophis & \footnotesize 2018 \newline 2029 & \footnotesize R, SR \newline R \T\B\\
    \hline
    \textbf{\footnotesize Lucy} & \footnotesize 2021 & \footnotesize 152830 Dinkinesh \newline 52246 Donaldjohanson \newline 3548 Eurybates \newline 15094 Polymele \newline 11351 Leucus \newline 21900 Orus \newline 617 Patroclus & \footnotesize 2023 \newline 2025 \newline 2027 \newline 2027 \newline 2028 \newline 2028 \newline 2033 & \footnotesize FB \newline FB \newline FB \newline FB \newline FB \newline FB \newline FB \T\B\\
    \hline
    \textbf{\footnotesize DART, LICIACube} & \footnotesize 2021 & \footnotesize 65803 Didymos & \footnotesize 2022 & \footnotesize I, FB \T\B\\
    \hline
    \textbf{\footnotesize Psyche} & \footnotesize 2023 & \footnotesize 16 Psyche & \footnotesize 2029 & R \T\B\\
    \hline
    \textbf{\footnotesize Hera, Milani, Juventas} & \footnotesize 2024 & \footnotesize 65803 Didymos & \footnotesize 2026 & \footnotesize R \T\B\\
    \hline
    \end{tabular}
    \\[5pt]
    \caption[Small bodies missions summary.]{Summary of missions to Solar System's small bodies. FB: fly-by, R: rendezvous, L: landing, SR: sample return, I: impact.}
    \label{tab:review}
\end{table}


\section{Approach and Proximity Phases}\label{Sec:Approach and Proximity Phases}

This section focuses on the approach and proximity phases of missions to small Solar System bodies. A detailed analysis of key mission identified two primary phases: the \textbf{Approach Phase} and the \textbf{Proximity Phase}, with the latter subdivided into the \textbf{Global Mapping} and the \textbf{Site-Specific Survey} sub-phases. For each phase, essential tasks derived from the previous literature review are outlined. The phase breakdown and associated objectives are as follows:

\begin{itemize}
    \item \textbf{Approach Phase}
    \begin{enumerate}
        \item Hazard survey: Observe the vicinity of the target asteroid, searching for possible natural satellites and/or a dusty environment that could pose risks to operations.
        \item Asteroid ephemeris update: Refine the description of the asteroid's orbital motion using the new data from the spacecraft's instruments.
        \item Point-source properties determination: Estimate the asteroid's physical and thermal properties relying on observations from the spacecraft's instruments.
    \end{enumerate}
    \item \textbf{Proximity Phase}
    \begin{itemize}
        \item \textbf{Global Mapping}
        \begin{enumerate}
            \item Initial physical characterization: Estimate the asteroid's shape, rotation state, mass, and generate an initial gravity model for the environment near the target.
            \item Navigation strategy switch: Transition from a star-based navigation strategy to a landmark-based one.
            \item Surface mapping: Map at least 80\% of the asteroid's surface using on-board cameras.
            \item Spectroscopic measurements: Conduct spectroscopic investigations using on-board instruments to estimate the target's thermal and infrared properties, constrain its internal composition, and refine models of radiative forces acting on the asteroid.
            \item Selection of candidate sites: Select up to twelve regions on the asteroid's surface for detailed observations in the next phase.
        \end{enumerate}
        \item \textbf{Site-Specific Survey}
        \begin{enumerate}
            \item Topography and spectral mapping: Refine the asteroid’s topography and spectral properties, and provide a detailed description of its composition.
            \item Refined gravity model: Use precise, continuous radiometric measurements to produce a more accurate gravity model for the asteroid.
            \item High-resolution observations: Reduce the spacecraft’s distance from the asteroid, yielding higher-resolution optical images to better map its surface.
            \item Site selection: For missions involving sample return, touch-down, or landing, detailed surface observations help identify the optimal site for these tasks.
        \end{enumerate}
    \end{itemize}
\end{itemize}

Each selected mission was analysed in detail to study the solutions employed during these phases. As a result, a unified set of requirements was established to compare the various approaches and generate initial requirements for the design of the approach and proximity phases of the mission under study. These \textbf{requirements} include:

\begin{itemize}
    \item \textbf{Distance}: The distance from the asteroid's barycenter (unless otherwise specified), expressed in terms of the target's radii.
    \item \textbf{Phase Angle}: The angle between the asteroid-Sun (or asteroid-Earth) and the asteroid-spacecraft position vectors.
    \item \textbf{Orbit Safety and Stability}: The design solutions ensuring that the trajectory remain safe and stable, avoiding escape or impact for extended durations.
    \item \textbf{Eclipse and Occultations}: While not essential, avoiding both Sun eclipses and Earth occultations is a nice to have requirement.
    \item \textbf{Maneuvering Time}: The minimum time between successive manoeuvres, expressed in days.
\end{itemize}

The missions selected for analysis are the most significant in terms of their rendezvous with small Solar System bodies. Each mission offers valuable insights into different phases of approach, proximity operations, and how to achieve various mission objectives.

The first mission chosen was \textbf{NEAR}, the first mission to rendezvous with a NEA. NEAR serves as an excellent baseline for missions that employ closed orbits around a large asteroid and those considering a landing of the entire spacecraft on the asteroid's surface.

\textbf{Rosetta} was also selected, despite it focuses on a comet rather than an asteroid. Rosetta's approach and proximity strategies, including mapping through closed orbits and fly-over maneuvers, are highly relevant, especially for missions that may end in a parking orbit for lander release and relay operations. 

\textbf{Hayabusa2} was added due to its unique mapping strategy based on a hovering technique. Even if this specific approach isn't adopted, it provides a valid baseline in case of changes in mission design that might favor similar strategies.

The \textbf{OSIRIS-REx} mission was selected due to its similarity to the mission under design, particularly in terms of the target's dimensions and the objectives of proximity operations. The mission design under study will begin by leveraging the solutions implemented by OSIRIS-REx, while attempting to reduce propellant consumption without compromising mission requirements' satisfaction.

Additionally, \textbf{Psyche} and \textbf{Hera} were incorporated into the analysis. Psyche was chose to examine proximity operations utilizing SEP, while Hera provides a useful reference in the event CubeSats are added to the mission or if the target asteroid turns out to be a binary system.

As previously mentioned, the selected missions were analyzed based on their approach, global mapping, and site-specific survey phases. The mission requirements were then compared to identify optimal strategies, creating a comprehensive dataset from which the initial requirements for the mission under design can be drawn.

The diversity among the selected missions is not a drawback but rather an advantage, as it broadens the range of potential solutions and increases the likelihood that any changes in the mission design will be adequately covered by the strategies and techniques utilized in these missions.


\subsection{Approach Phase}\label{Ssec:Approach Phase}

The Approach Phase solutions are summarized in Table~\ref{tab:approach}, highlighting key trends and strategies. From this table, is evident that there is a \textbf{starting distance} at which the first images of the target are acquired by on-board cameras, and an \textbf{ending distance} just outside of the target's sphere of influence (SOI), yet close enough initiate proximity operations. 

Regarding the \textbf{Sun phase angle}, the most effective strategy is to maintain a value under 70° throughout the entire approach phase, primarily to avoid \textbf{Sun eclipse}. Furthermore, it is beneficial to also avoid \textbf{Earth occultations}, enabling continuous communication with the spacecraft. \textbf{Orbit safety and stability} are not critical at this stage, as the final approach distance lies outside the target's SOI. Nevertheless, the use of hyperbolic trajectories appears to be the most suitable approach. 

Lastly, an \textbf{interval between successive manoeuvres} of three to seven days seems to provide a balanced solution, allowing sufficient time for orbit corrections and trajectory refinement while maintaining operational efficiency.


\subsection{Global Mapping}\label{Ssec:Global Mapping}

As shown in Table~\ref{tab:globalmapping}, the Global Mapping phase exhibits less uniformity in solutions compared to the approach phase (Table~\ref{tab:approach}), as the chosen strategy is highly dependent on the characteristics of the target. Nevertheless, general trends can still be observed and a comparison can be performed. The operational \textbf{distance} typically ranges from 10 to 40 target radii, with lower excursions down to around 4 radii in the case of closed orbits around a low-mass bodies, such as in the OSIRIS-REx mission. 

In this phase, \textbf{orbit safety and stability} become critical due to the spacecraft's closer distance form the target and potential interactions with its gravitational field. Hyperbolic trajectories appear to be the most promising solutions, although SSTO and retrograde polar orbits are also viable options, depending on the specific mission objectives.

A key requirement linked to trajectory selection is the \textbf{Sun phase angle}, which is generally limited up to 70° for observations along hyperbolic or retrograde polar orbits. However, this angle can increase to 85°-95° if SSTO is employed. For specific observation requirements, the Sun phase angle might be restricted further, such for Hera's CubeSat Milani, where an phase angle of 5°-25° during observations is preferred.

Regarding \textbf{eclipse and occultation}, Sun eclipses are inherently avoided due to the chosen trajectories, while avoiding Earth occultations would be beneficial for maintaining continuous communication with the spacecraft. Finally, the \textbf{time interval between successive manoeuvres} during this phase is typically fixed between one to four days.


\subsection{Site-Specific Survey}\label{Ssec: Site-Specific Survey}

The solutions for the Site-Specific Survey phase are summarized in Table~\ref{tab:sitespecificsurvey}, reflecting the differences already noted in Table~\ref{tab:globalmapping}. One key observation is that the \textbf{distances} during this phase fall within the range of 1.8 to 6 target's radii, placing the spacecraft within the target’s gravitational SOI. This introduces significant challenges in ensuring \textbf{orbit safety and stability}. To address this, hyperbolic loop trajectories are often used. These trajectories allow the spacecraft to enter the SOI for close approaches before escaping, thereby minimizing prolonged exposure to the target's gravitational field. Alternatively, closed orbits stabilized by the Sun or through gravitational resonances, such as SSTO or retrograde polar orbits, are feasible options. Retrograde orbits, particularly polar ones, are often preferable due to their ability to support high-resolution observations.

The \textbf{Sun phase angle}, influenced by the choice of the trajectory, is typically restricted to 70° for observations during hyperbolic loop trajectories or retrograde orbits. If SSTO orbits are employed, the phase angle may increase in the range of 85°-95°. For consistency with earlier phases, \textbf{Sun eclipses} and \textbf{Earth occultations} should be avoided whenever possible to ensure uninterrupted solar power and Earth communication.

Finally, as observed in the Global Mapping phase, the \textbf{interval between successive maneuvers} during the Site-Specific Survey phase should remain within the range of one to four days to allow for careful planning and adjustment.


\subsection{Requirements for Approach and Proximity Phases}\label{Ssec:Requirements for Approach and Proximity Phases}

The analysis conducted above has defined the requirements and solutions for both Approach and Proximity Phases. The results are summarized in Table~\ref{tab:summary}. For the mission under design, the \textbf{Approach Phase} is initiated at the point of the first on-board optical acquisition of the asteroid and proceeds until the spacecraft reaches a distance of 40 to 80 target's radii. During this phase, the Sun phase angle must remain below 70° to avoid Sun eclipses and guarantee a good illumination of the target; Earth occultations should also be avoided to maintain continuous communication. Maneuvers are planned with an interval down to three to four days, concluding with a hyperbolic insertion if necessary.

After the Approach Phase, the \textbf{Global Mapping} begins, with two distinct sets of operations. The first set is carried out at a distance of 25 to 40 asteroid's radii to gather initial estimates of the target’s physical properties. In the second set of operations, the spacecraft reduces its distance to between 10 and 25 asteroid's radii to refine previous estimates and map the surface of the target in more detail.

The \textbf{Site-Specific Survey} follows the Global Mapping, with the key difference being the distance of the operations. During this phase, the spacecraft operates between 3 and 6 asteroid's radii to perform high-resolution mapping and refine physical property estimates. Further, it reduces the distance to within 1.8 to 3 asteroid's radii for detailed observations of selected surface sites and specific scientific measurements. 

For Global Mapping, hyperbolic loop trajectories are generally preferred, as they are outside the asteroid's SOI, providing safe and stable operations. If the target has strong gravitational influence and a large SOI, SSTO or retrograde orbits may also be considered. The same considerations apply to the Site-Specific Survey, but hyperbolic loops are used not due to low gravity but to minimize residence time inside the asteroid's SOI, which dynamics can become non-linear, as in the case of binary asteroids or highly non-spherical bodies.

The Sun phase angle depends on the selected orbit. For SSTO orbits, the Sun phase angle should be between 85° and 95°, while for hyperbolic loop trajectories or retrograde orbits, it should not exceed 70° during observation segments. Specific mission requirements may impose additional constraints, such as for Hera’s CubeSat Milani, which restricted the Sun phase angle to ranges of 0°-10°, 5°-25°, and 40°-70° for specific observations. Earth occultations should preferably be avoided for consistent communication, while Sun eclipses are naturally avoided by using SSTO or hyperbolic loops. However, if retrograde orbits are employed, Sun eclipses should be limited to 65 minutes.

Finally, the minimum interval between successive maneuvers during both the global mapping and site-specific survey phases should be no less than one to two days to allow for careful planning and trajectory corrections.

\begin{landscape}
\begin{table}[ht!]
\centering
\renewcommand{\arraystretch}{1.5}
\begin{tabular}{| >{\raggedright\arraybackslash}m{3cm} | >{\raggedright\arraybackslash}m{4cm} | >{\raggedright\arraybackslash}m{3.5cm} | >{\raggedright\arraybackslash}m{3.5cm} | >{\raggedright\arraybackslash}m{3.5cm} | >{\raggedright\arraybackslash}m{3.5cm} |}
\hline
\rowcolor{bluepoli!40}
\textbf{Mission} & \textbf{Distance \newline [Target Radii]} & \textbf{Phase Angle \newline [deg]} & \textbf{Orbit Safety and Stability} & \textbf{Eclipse and Occultation} & \textbf{Maneuvering Time [days]} \\
\hline\hline
\textbf{NEAR} & \footnotesize Insertion in a 60 radii parking orbit through an hyperbolic trajectory with pericentre of 120 radii & \footnotesize Sun phase angle equal to 0° at insertion point & \footnotesize Hyperbolic trajectory & \footnotesize No Sun eclipse \newline No Earth occultation & \footnotesize 7 days between successive manoeuvres \\
\hline
\textbf{Rosetta} & \footnotesize From 250000 radii until insertion point at 40 radii & \footnotesize Sun phase angle always less than 70° & \footnotesize Hyperbolic trajectory & \footnotesize No Sun eclipse \newline No Earth occultation & \footnotesize 3 days between successive manoeuvres \\
\hline
\textbf{Hayabusa2} & \footnotesize Target distance of 46 radii from the sub-Earth surface point & \footnotesize Earth phase angle always about equal to 0° & \footnotesize Station-keeping around an hovering position & \footnotesize No Sun eclipse \newline Yes Earth occultation & -- \\
\hline
\textbf{OSIRIS-REx} & \footnotesize From about 600000 radii until an altitude of about 80 radii & \footnotesize Sun phase angle always less than 70° & \footnotesize Hyperbolic trajectory out of asteroid SOI & \footnotesize No Sun eclipse \newline \footnotesize No Earth occultation & \footnotesize 7 days between successive manoeuvres \\
\hline
\textbf{Psyche} & \footnotesize From about 600 radii to 7.2 radii surface altitudes through a low-thrust trajectory & -- & -- \footnotesize (pay attention to spin-orbit resonance) & \footnotesize No Sun eclipse \newline No Earth occultation & -- \\
\hline
\textbf{Hera} & \footnotesize Target distance from the primary of 35.5 radii in the Sun-side of the system & \footnotesize Sun phase angle always less than 70° & \footnotesize Hyperbolic trajectory out of the asteroid SOI & \footnotesize No Sun eclipse \newline No Earth occultation & \footnotesize 3-4 days between successive manoeuvres \\
\hline
\end{tabular}
\\[5pt]
\caption[Approach requirements.]{Approach requirements from a selection of missions to small bodies.}
\label{tab:approach}
\end{table}
\end{landscape}

\begin{landscape}
\begin{table}[ht!]
\centering
\renewcommand{\arraystretch}{1.5}
\begin{tabular}{| >{\raggedright\arraybackslash}m{3cm} | >{\raggedright\arraybackslash}m{4cm} | >{\raggedright\arraybackslash}m{3.5cm} | >{\raggedright\arraybackslash}m{3.5cm} | >{\raggedright\arraybackslash}m{3.5cm} | >{\raggedright\arraybackslash}m{3.5cm} |}
\hline
\rowcolor{bluepoli!40}
\textbf{Mission} & \textbf{Distance \newline [Target Radii]} & \textbf{Phase Angle \newline [deg]} & \textbf{Orbit Safety and Stability} & \textbf{Eclipse and Occultation} & \textbf{Maneuvering Time [days]} \\
\hline\hline
\textbf{NEAR} & \footnotesize From a mean orbit of 24 radii until one of 12 radii, gradually & \footnotesize Sun phase angle less than 70° during observations & \footnotesize Retrograde closed orbits, preferable if polar & \footnotesize No Sun eclipse \newline No Earth occultation & \footnotesize 7 days between successive manoeuvres \\
\hline
\textbf{Rosetta} & \footnotesize Orbit in the distance range of 10-25 radii from the surface & \footnotesize Sun phase angle less than 70° during observations & \footnotesize Nearly polar closed retrograde orbits, in the terminator plane  & \footnotesize No Sun eclipse \newline No Earth occultation & \footnotesize 1 day between successive manoeuvres \\
\hline
\textbf{Hayabusa2} & \footnotesize Hovering at a distance of 46 radii from the sub-Earth surface point & \footnotesize Earth phase angle always about 0° & \footnotesize Station-keeping around an hovering position & \footnotesize No Sun eclipse \newline Yes Earth occultation & -- \\
\hline
\textbf{OSIRIS-REx} & \footnotesize Closest approaches with altitude of 27 radii and in the range of 13.3-19 radii; \newline Orbit with an altitude in the range of 3.8-5.7 radii & \footnotesize Sun phase angle less than 95° and then below 70°; \newline During orbit within 85° and 95° & \footnotesize Hyperbolic trajectory; \newline SSTO within the SOI & \footnotesize No Sun eclipse \newline \footnotesize No Earth occultation & \footnotesize 2 days between successive manoeuvres \\
\hline
\textbf{Psyche} & \footnotesize Orbit at about 7.2 radii and 3.6 radii altitudes & -- & \footnotesize Always on closed polar orbits & \footnotesize Yes Sun eclipse, limited to 65 minutes \newline No Earth occultation & -- \\
\hline
\textbf{Hera} & \footnotesize Closest approaches with distances in the ranges of 50-75 and 20-50 radii & \footnotesize Sun phase angle always less than 70° & \footnotesize Hyperbolic loop trajectories out of the asteroid SOI & \footnotesize No Sun eclipse \newline No Earth occultation & \footnotesize 3-4 days between successive manoeuvres \\
\hline
\textbf{Milani} \newline \footnotesize (Hera's CubeSat) & \footnotesize Closest approaches in the range of 22-28 radii & \footnotesize Sun phase angle less than 70°; observe in the range of 5°-25° & \footnotesize Hyperbolic loop trajectory out of the asteroid SOI & \footnotesize No Sun eclipse \newline No Earth occultation & \footnotesize 3-4 days between successive manoeuvres \\
\hline
\textbf{Juventas} \newline \footnotesize (Hera's CubeSat) & \footnotesize Orbit of about 8 radii & \footnotesize Sun phase angle in the range of 85°-95° & \footnotesize SSTO inside the asteroid SOI & \footnotesize No Sun eclipse \newline No Earth occultation & \footnotesize 3-4 days between successive manoeuvres \\
\hline
\end{tabular}
\\[5pt]
\caption[Global mapping requirements.]{Global mapping requirements from a selection of missions to small bodies.}
\label{tab:globalmapping}
\end{table}
\end{landscape}

\begin{landscape}
\begin{table}[ht!]
\centering
\renewcommand{\arraystretch}{1.5}
\begin{tabular}{| >{\raggedright\arraybackslash}m{3cm} | >{\raggedright\arraybackslash}m{4cm} | >{\raggedright\arraybackslash}m{3.5cm} | >{\raggedright\arraybackslash}m{3.5cm} | >{\raggedright\arraybackslash}m{3.5cm} | >{\raggedright\arraybackslash}m{3.5cm} |}
\hline
\rowcolor{bluepoli!40}
\textbf{Mission} & \textbf{Distance \newline [Target Radii]} & \textbf{Phase Angle \newline [deg]} & \textbf{Orbit Safety and Stability} & \textbf{Eclipse and Occultation} & \textbf{Maneuvering Time [days]} \\
\hline\hline
\textbf{NEAR} & \footnotesize Fly-over from orbits of 6, 4, and 2.4 radii & \footnotesize Sun phase angle less than 70° during observations & \footnotesize Retrograde closed orbits, preferable if polar & \footnotesize No Sun eclipse \newline No Earth occultation & \footnotesize 7 days between successive manoeuvres \\
\hline
\textbf{Rosetta} & \footnotesize Fly-over from orbits with mean distance in the range of 2-5 radii & \footnotesize Sun phase angle less than 70° during observations & \footnotesize Nearly polar closed retrograde orbits, in the terminator plane  & \footnotesize No Sun eclipse \newline No Earth occultation & \footnotesize 0.5 days between successive manoeuvres \\
\hline
\textbf{Hayabusa2} & \footnotesize Hovering at distance within 2.3-11.5 radii from sub-Earth surface point & \footnotesize Earth phase angle always about 0° & \footnotesize Station-keeping around an hovering position & \footnotesize No Sun eclipse \newline Yes Earth occultation & -- \\
\hline
\textbf{OSIRIS-REx} & \footnotesize Orbit with an altitude of about 3.8 radii; \newline Closest approaches of 0.86 and 2 radii altitude & \footnotesize Sun phase angle within 85° and 95°; \newline In the range of 40°-70° during fly-bys  & \footnotesize SSTO within the SOI; \newline Hyperbolic trajectory within the SOI & \footnotesize No Sun eclipse \newline \footnotesize No Earth occultation & \footnotesize 2 days between successive manoeuvres \\
\hline
\textbf{Psyche} & \footnotesize Orbit at about 2.5 radii altitude; Orbit at about 1.75 radii altitude & --\footnotesize ; \newline Sun phase angle within 85° and 95° & \footnotesize Closed polar orbit; Closed retrograde equatorial orbit & \footnotesize Yes Sun eclipse, limited to 65 minutes \newline No Earth occultation & -- \\
\hline
\textbf{Hera} & \footnotesize Closest approaches in the ranges of 2.5-10 radii & \footnotesize Sun phase angle always less than 90° & \footnotesize Hyperbolic loop trajectory & \footnotesize No Sun eclipse \newline No Earth occultation & \footnotesize 3-4 days between successive manoeuvres \\
\hline
\textbf{Milani} \newline \footnotesize (Hera's CubeSat) & \footnotesize Closest approaches in the ranges of 3.1-6.8 and 1.5-3.5 radii & \footnotesize Sun phase angle in the ranges of 30°-60° and 0°-10° for observations & \footnotesize Hyperbolic loop trajectory & \footnotesize No Sun eclipse \newline No Earth occultation & \footnotesize 3-4 days between successive manoeuvres \\
\hline
\textbf{Juventas} \newline \footnotesize (Hera's CubeSat) & \footnotesize Orbit of about 4 radii & \footnotesize Sun phase angle in the range of 85-95° & \footnotesize SSTO inside the asteroid SOI & \footnotesize No Sun eclipse \newline No Earth occultation & \footnotesize 3-4 days between successive manoeuvres \\
\hline
\end{tabular}
\\[5pt]
\caption[Site-specific survey requirements.]{Site-specific survey requirements from a selection of missions to small bodies.}
\label{tab:sitespecificsurvey}
\end{table}
\end{landscape}

\begin{landscape}
\begin{table}[ht!]
\centering
\renewcommand{\arraystretch}{1.5}
\begin{tabular}{| >{\raggedright\arraybackslash}m{4cm} | >{\raggedright\arraybackslash}m{4cm} | >{\raggedright\arraybackslash}m{7cm} | >{\raggedright\arraybackslash}m{7cm} |}
\hline
\rowcolor{bluepoli!40}
\textbf{Requirements} & \textbf{Approach} & \textbf{Global Mapping} & \textbf{Site-Specific Survey} \\
\hline\hline
\textbf{Distance from Target \newline [Target Radii]} & From first on-board optical acquisition distance to 80-40 radii from target & In the range of 25-40 radii to estimate target properties (mass, shape, gravity, spin state); \newline In the range of 10-25 radii to refine estimates of target properties and map the target (thermal/infrared/interior properties) & In the range of 3-6 radii to refine estimates of target properties and map the target with high-resolution; \newline In the range of 1.8-3 radii to map selected sites and perform scientific observations \\
\hline
\textbf{Sun Phase Angle for Observations [deg]} & Always less than 70° & In the range of 85°-95° if SSTO, otherwise less than 70°; \newline Specific requirements can be in the range of 5°-25° or less than 95° & In the range of 85°-95° if SSTO, otherwise less than 70°; \newline Specific requirements can be in the range of 0°-10°, 5°-25°, 40°-70°, or less than 90° \\
\hline
\textbf{Orbit Safety and Stability} & Insertion through an hyperbolic orbit after a gradual reduction of relative distance and velocity & SSTO; \newline Hyperbolic loop trajectories; \newline Retrograde closed orbit, preferable if polar & SSTO; \newline Hyperbolic loop trajectories; \newline Retrograde closed orbit, preferable if polar \\
\hline
\textbf{Eclipse and Occultation} & No Sun eclipse \newline No Earth occultation & No Sun eclipse if hyperbolic loop or SSTO, otherwise limited to 65 minutes \newline No Earth occultation & No Sun eclipse if hyperbolic loop or SSTO, otherwise limited to 65 minutes \newline No Earth occultation \\
\hline
\textbf{Maneuvering Time [days]} & 3-4 days between successive manoeuvres & 1-2 days between successive manoeuvres & 1-2 days between successive manoeuvres \\
\hline
\end{tabular}
\\[5pt]
\caption[Requirements summary.]{Requirements summary from a selection of missions to small bodies.}
\label{tab:summary}
\end{table}
\end{landscape}

\end{document}